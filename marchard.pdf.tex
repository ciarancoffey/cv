\documentclass[]{article}
\usepackage{lmodern}
\usepackage{amssymb,amsmath}
\usepackage{ifxetex,ifluatex}
\usepackage{fixltx2e} % provides \textsubscript
\ifnum 0\ifxetex 1\fi\ifluatex 1\fi=0 % if pdftex
  \usepackage[T1]{fontenc}
  \usepackage[utf8]{inputenc}
\else % if luatex or xelatex
  \ifxetex
    \usepackage{mathspec}
    \usepackage{xltxtra,xunicode}
  \else
    \usepackage{fontspec}
  \fi
  \defaultfontfeatures{Mapping=tex-text,Scale=MatchLowercase}
  \newcommand{\euro}{€}
\fi
% use upquote if available, for straight quotes in verbatim environments
\IfFileExists{upquote.sty}{\usepackage{upquote}}{}
% use microtype if available
\IfFileExists{microtype.sty}{%
\usepackage{microtype}
\UseMicrotypeSet[protrusion]{basicmath} % disable protrusion for tt fonts
}{}
\ifxetex
  \usepackage[setpagesize=false, % page size defined by xetex
              unicode=false, % unicode breaks when used with xetex
              xetex]{hyperref}
\else
  \usepackage[unicode=true]{hyperref}
\fi
\hypersetup{breaklinks=true,
            bookmarks=true,
            pdfauthor={},
            pdftitle={},
            colorlinks=true,
            citecolor=blue,
            urlcolor=blue,
            linkcolor=magenta,
            pdfborder={0 0 0}}
\urlstyle{same}  % don't use monospace font for urls
\setlength{\parindent}{0pt}
\setlength{\parskip}{6pt plus 2pt minus 1pt}
\setlength{\emergencystretch}{3em}  % prevent overfull lines
\providecommand{\tightlist}{%
  \setlength{\itemsep}{0pt}\setlength{\parskip}{0pt}}
\setcounter{secnumdepth}{0}

\date{}
\usepackage[top=1in, bottom=1in, left=1in, right=1in]{geometry}
\usepackage{tgpagella}
\ifxetex
  \usepackage{xeCJK}
  \setCJKmainfont{WenQuanYi Zen Hei}
  %\setmainfont[Mapping=tex-text]{TeX Gyre Pagella}
\fi

\usepackage{titlesec}
\titleformat{\section}{\huge\bfseries}{\thesection}{1em}{}
\titlespacing{\section}{0pt}{-2em}{1em}
\titleformat{\subsection}{\large\bfseries\MakeUppercase}{\thesubsection}{1em}{}

%\usepackage{multicol}
\pagestyle{empty}
\hyphenation{Media-Wiki}

\renewcommand{\labelitemi}{}
\renewcommand{\labelitemii}{\raise .5ex\hbox{\tiny$\blacktriangleright$}}

% from http://tex.stackexchange.com/a/29796/16139
\newsavebox{\zerobox}
\newenvironment{nospace}
  {\par\edef\theprevdepth{\the\prevdepth}\nointerlineskip
   \setbox\zerobox=\vtop to 0pt\bgroup
   \hrule height0pt\kern\dimexpr\baselineskip-\topskip\relax
  }
  {\par\vss\egroup\ht\zerobox=0pt \wd\zerobox=0pt \dp\zerobox=0pt
   \box\zerobox}

% Redefines (sub)paragraphs to behave more like sections
\ifx\paragraph\undefined\else
\let\oldparagraph\paragraph
\renewcommand{\paragraph}[1]{\oldparagraph{#1}\mbox{}}
\fi
\ifx\subparagraph\undefined\else
\let\oldsubparagraph\subparagraph
\renewcommand{\subparagraph}[1]{\oldsubparagraph{#1}\mbox{}}
\fi

\begin{document}

\begin{nospace}\begin{flushright}
\vspace{-2em}\href{mailto:marchard@gmail.com}{marchard@gmail.com}

CORU Membership Number: DI008266
\end{flushright}\end{nospace}

\section{Michelle Archard}\label{michelle-archard}

\subsection{Work Experience}\label{work-experience}

\begin{itemize}
\item
  \textbf{St.~Vincents University Hospital} (Dublin)

  \emph{Dietitian}, Nov 2016 - Present

  Dietitian

  \begin{itemize}
  \tightlist
  \item
    Surgical Dietitian covering adult inpatients including vascular,
    urology, gynacology, plastics, ENT and oncology specialities
    (0.7WTE).
  \item
    Additionaly cover ICU seeing a variety of patients on continuous
    dialysis +/- ventilated / intubated (0.3WTE).
  \item
    Involved in MDT Major Limb Amputee Pathway including contributing to
    patient pathway booklet for SVUH.
  \item
    See patients requiring variety of dietetic interventions including
    food fortification, renal diet, oral nutritional supplements,
    enteral feeding and total parenteral nutrition.
  \item
    Liaise with community dietitians and other hospitals regarding
    patient transfer / handover.
  \end{itemize}
\item
  \textbf{North Middlesex University Hospital} (London N18 1QX)

  \emph{Band 6 Paediatric Dietitian}, June 2015 - Nov 2016

  \begin{itemize}
  \tightlist
  \item
    Worked as a full time Paediatric Dietitian as part of a four member
    team in North London which is a secondary healthcare centre. My team
    members included specialist Neonatal, Diabetes and Allergy
    Dietitians whom I liaised with regarding complex patients and also
    have opportunity to shadow and appreciate the value of peer support.
  \item
    I screened and assessed patients in these areas when my team may be
    on annual leave and when weekend working.
  \item
    I have always had a keen interest in dietetics and would enjoy
    building on my transferable dietetic skills and working with adults.
  \item
    Evidence-based strategies were employed in my formulation of care
    planning regarding weight management, nutrition support, food
    allergy, T1DM, NG NJ, PEG and PEG-J feeding.
  \item
    The practice involved multidisciplinary meetings and liaising with
    patients and their parents / carers, GPs, consultants, nurses and
    other allied health professionals, particularly Speech and Language
    Therapy.
  \item
    I regularly attend ward rounds with the medical team where I screen
    patients and also participate in discharge planning meetings when
    appropriate. On a daily basis I clearly, concisely and effectively
    communicate using appropriate methods with patients, carers and
    other professionals while ensuring confidentiality is maintained.
  \item
    Based in an area with a diverse range of ethnicities where I
    encounter people with varying backgrounds and cultures. I enjoyed
    the challenge of adapting my teaching and education skills in order
    to clearly, effectively and safely convey important and relevant
    information to patients, parents, family members and carers to
    ensure the best possible outcome for service users.
  \item
    Weekend working involved close liaison with the medical team and
    ward staff and ensuring effective handover and communication with my
    colleagues for safe continuing care of service users.
  \end{itemize}
\end{itemize}

\pagebreak

\begin{itemize}
\item
  \textbf{Dumfries\&Galloway Royal Infirmary} (Dumfries, Scotland DG1
  4AP)

  \emph{Band 5 Paediatric Dietitian}, Jan 2014 - Feb 2015

  My first post as a newly qualified Dietitian was as part of a
  four-member team in a hospital in Scotland.

  \begin{itemize}
  \tightlist
  \item
    A large proportion of my case load involved organising outpatient
    clinics where I assessed, educated and advised children and their
    families on an individual basis.
  \item
    The role covered a large geographical area where I facilitated
    outpatient clinics within and out with the hospital premises.
  \item
    This post also involved liaising with the tertiary centre in Glasgow
    regarding renal, hepatology, cystic fibrosis and gastroenterology
    patients.
  \item
    My main role was as the ward Dietitian where I screened new patients
    and also known community patients whom I discussed with my
    colleagues to review and update plans.
  \item
    I supported patients requiring home enteral feeding and long-term
    follow up.
  \end{itemize}
\end{itemize}

\subsection{Clinical Experience}\label{clinical-experience}

\begin{itemize}
\item
  \textbf{B Placement: Glan Clywd Hospital} (Rhyl, Wales), Jan - Apr
  2012
\item
  \textbf{C Placement: University Hospital Wales} (Cardiff, Wales), Aug
  - Oct 2012
\end{itemize}

\subsection{Education}\label{education}

\begin{itemize}
\item
  \textbf{Cardiff Metropolitan University} (Cardiff)

  \emph{BSc Hons 1st, Human Nutrition and Dietetics} 2009-2013
\item
  \textbf{NUI Galway} (Galway)

  \emph{BEng 2:2, Biomedical Engineering} 2003-2007
\end{itemize}

\subsection{Hobbies}\label{hobbies}

\begin{itemize}
\tightlist
\item
  Clinical dietetics
\item
  Movies
\item
  Swimming
\item
  Cycling
\item
  Reading
\item
  Roller-blading
\end{itemize}

\end{document}
